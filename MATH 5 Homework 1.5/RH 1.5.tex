% !root= RH 1.3.tex
\documentclass{article}
\usepackage{tikz}
\usepackage{pgfplots}
\usepackage{setspace}
\usepackage{units}
\usepackage{graphicx}
\usepackage{amsopn}
\usepackage{bbding}
\usepackage{amsmath}
\usepackage{hyperref}
\usepackage{cancel}
\usepackage{etoolbox}
\usepackage{enumitem}
\usepackage{gensymb}
\usepackage{amssymb}
\usepackage{multicol}
\usepackage{numerica}
\usepackage[top=0.7in, left=0.8in, right=0.8in, bottom=1in]{geometry}
\AtBeginEnvironment{document}{\everymath{\displaystyle}}
\pagestyle{plain}
\title{RH 1.4}
\author{MATH 5, Jones}
\date{Tejas Patel}
\begin{document}
\maketitle
\section*{Refrigerator Homework}
\subsection*{5}
$
\begin{bmatrix}1&3&1&0\\-4&-9&2&0\\0&-3&-6&0\end{bmatrix} \rightarrow R_2+=4R_1 \rightarrow \begin{bmatrix}1&3&1&0\\0&3&6&0\\0&-3&-6&0\end{bmatrix}\rightarrow R_3+=R_2, R_2/=3\rightarrow \begin{bmatrix}1&3&1&0\\0&1&2&0\\0&0&0&0\end{bmatrix}
\\[0.1in] R_1-=3R_2 \rightarrow \begin{bmatrix}1&0&-5&0\\0&1&2&0\\0&0&0&0\end{bmatrix}
\rightarrow x_1=5x_3 \quad x_2=-2x_3 \quad x_3=x_3 
$  \boxed{
    \begin{bmatrix} x_1 \\ x_2 \\ x_3 \end{bmatrix} =x_3 \begin{bmatrix} 5 \\ -2 \\ 1 \end{bmatrix}}
\subsection*{7}
$\begin{bmatrix}
    1&3&-3&7&0\\0&1&-4&5&0
\end{bmatrix}\rightarrow R_1-=3R_2 \rightarrow \begin{bmatrix}
    1&0&9&-8&0\\0&1&-4&5&0
\end{bmatrix}$  $x_3=s,\; x_4=t$
\\[0.1in]$x_1=8t-9s,\;x_2=4s-5t,\;x_3=s,\;x_4=t$
\boxed{\begin{bmatrix} x_1 \\ x_2 \\ x_3 \\ x_4 \end{bmatrix} =
x_3 \begin{bmatrix} 9 \\ -4 \\ 1 \\ 0 \end{bmatrix} + x_4 \begin{bmatrix} 8 \\ -5 \\ 0 \\ 1 \end{bmatrix}}
\subsection*{20}
$\begin{bmatrix}1&3&-5&4\\1&4&-8&7\\-3&-7&9&-6\end{bmatrix}\rightarrow R_2-=R_1,\;R_3+=3R_1\rightarrow\begin{bmatrix}1&3&-7&4 \\ 0&1&-3&3\\ 0&2&-6&6\end{bmatrix}\\R_3-=2R_2,\;R_1-=3R_2 \rightarrow \begin{bmatrix}1&0&2&-5\\0&1&-3&3\\0&0&0&0\end{bmatrix} x_3=t
\\[0.1in]x_1=-5-2t,\;x_2=3+3t$ \boxed{\begin{bmatrix}x_1\\x_2\\x_3\end{bmatrix} = \begin{bmatrix}-5\\3\\0 \end{bmatrix}+ x_3\begin{bmatrix}-2\\3\\1 \end{bmatrix}}
\subsection*{23}
Line is y=-x-2
so the equation is $x=\begin{bmatrix}0\\-2\end{bmatrix} + t\begin{bmatrix}1 \\ -1\end{bmatrix}$
\subsection*{28}
False: Nontrivial means at least 1 is nonzero, not all of them
\subsection*{40}
No. The origin is where $b=0$ in all dimensions, so $b$ would need to equal 0

\section{Computer Homework}
\subsection{}
\subsection{}
\subsection{}
\subsection{}
\subsection{}
\subsection{}
\subsection{}
\subsection{}
\subsection{}
\subsection{}

\end{document}