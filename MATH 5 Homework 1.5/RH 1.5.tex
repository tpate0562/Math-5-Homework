% !root= RH 1.3.tex
\documentclass{article}
\usepackage{tikz}
\usepackage{pgfplots}
\usepackage{setspace}
\usepackage{units}
\usepackage{graphicx}
\usepackage{amsopn}
\usepackage{bbding}
\usepackage{amsmath}
\usepackage{hyperref}
\usepackage{cancel}
\usepackage{etoolbox}
\usepackage{enumitem}
\usepackage{gensymb}
\usepackage{amssymb}
\usepackage{multicol}
\usepackage{numerica}
\usepackage[top=0.7in, left=0.8in, right=0.8in, bottom=1in]{geometry}
\AtBeginEnvironment{document}{\everymath{\displaystyle}}
\pagestyle{plain}
\title{RH 1.5}
\author{MATH 5, Jones}
\date{Tejas Patel}
\begin{document}
\maketitle
\section*{Refrigerator Homework}
\subsection*{5}
$
\begin{bmatrix}1&3&1&0\\-4&-9&2&0\\0&-3&-6&0\end{bmatrix} \rightarrow R_2+=4R_1 \rightarrow \begin{bmatrix}1&3&1&0\\0&3&6&0\\0&-3&-6&0\end{bmatrix}\rightarrow R_3+=R_2, R_2/=3\rightarrow \begin{bmatrix}1&3&1&0\\0&1&2&0\\0&0&0&0\end{bmatrix}
\\[0.1in] R_1-=3R_2 \rightarrow \begin{bmatrix}1&0&-5&0\\0&1&2&0\\0&0&0&0\end{bmatrix}
\rightarrow x_1=5x_3 \quad x_2=-2x_3 \quad x_3=x_3 
$  \boxed{
    \begin{bmatrix} x_1 \\ x_2 \\ x_3 \end{bmatrix} =x_3 \begin{bmatrix} 5 \\ -2 \\ 1 \end{bmatrix}}
\subsection*{7}
$\begin{bmatrix}
    1&3&-3&7&0\\0&1&-4&5&0
\end{bmatrix}\rightarrow R_1-=3R_2 \rightarrow \begin{bmatrix}
    1&0&9&-8&0\\0&1&-4&5&0
\end{bmatrix}$  $x_3=s,\; x_4=t$
\\[0.1in]$x_1=8t-9s,\;x_2=4s-5t,\;x_3=s,\;x_4=t$
\boxed{\begin{bmatrix} x_1 \\ x_2 \\ x_3 \\ x_4 \end{bmatrix} =
x_3 \begin{bmatrix} 9 \\ -4 \\ 1 \\ 0 \end{bmatrix} + x_4 \begin{bmatrix} 8 \\ -5 \\ 0 \\ 1 \end{bmatrix}}
\subsection*{20}
$\begin{bmatrix}1&3&-5&4\\1&4&-8&7\\-3&-7&9&-6\end{bmatrix}\rightarrow R_2-=R_1,\;R_3+=3R_1\rightarrow\begin{bmatrix}1&3&-7&4 \\ 0&1&-3&3\\ 0&2&-6&6\end{bmatrix}\\R_3-=2R_2,\;R_1-=3R_2 \rightarrow \begin{bmatrix}1&0&2&-5\\0&1&-3&3\\0&0&0&0\end{bmatrix} x_3=t
\\[0.1in]x_1=-5-2t,\;x_2=3+3t$ \boxed{\begin{bmatrix}x_1\\x_2\\x_3\end{bmatrix} = \begin{bmatrix}-5\\3\\0 \end{bmatrix}+ x_3\begin{bmatrix}-2\\3\\1 \end{bmatrix}}
\subsection*{23}
Line is y=-x-2 so the equation is $x=\begin{bmatrix}0\\-2\end{bmatrix} + t\begin{bmatrix}1 \\ -1\end{bmatrix}$
\subsection*{28}
False: Nontrivial means at least 1 is nonzero, not all of them
\subsection*{40}
No. The origin is where $b=0$ in all dimensions, so $b$ would need to equal 0

\section{Computer Homework}
\subsection*{1}
Determine if the system has a nontrivial solution. Try to use as few row operations as possible
$2x_1-9x_2+11x_3=0 \\ -2x_1-11x_2+4x_3=0\\4x_1+2x_2+7x_3=0$
\\ \boxed{$The system has a nontrivial solution$}
\subsection*{2}
Describe all solution of $A\mathbf{x}=\mathbf{0}$ in parametric vector form, where A is row equivalent to the given matrix.
$\begin{bmatrix}1&4&-3&5\\0&1&-2&6\end{bmatrix}$
\\ \boxed{\mathbf{x}=x_3\begin{bmatrix}-5\\2\\1\\0\end{bmatrix}+x_4\begin{bmatrix}19\\-6\\0\\1\end{bmatrix}}
\subsection*{3}
Describe all solution of $A\mathbf{x}=\mathbf{0}$ in parametric vector form, where A is row equivalent to the given matrix.
$\begin{bmatrix}1&2&0&-3\\4&8&0&-12\end{bmatrix}$
\\ \boxed{\mathbf{x}=x_2\begin{bmatrix}-2\\1\\0\\0\end{bmatrix}+x_3x_2\begin{bmatrix}0\\0\\1\\0\end{bmatrix} + x_4 x_2\begin{bmatrix}3\\0\\0\\1\end{bmatrix}}
\subsection*{4}
Describe all solution of $A\mathbf{x}=\mathbf{0}$ in parametric vector form, where A is row equivalent to the given matrix.
$\begin{bmatrix}1&4&3&-4&0&-1\\0&0&1&0&0&-8\\0&0&0&1&0&-3\\0&0&0&0&0&0\end{bmatrix}$
\\ \boxed{\mathbf{x}=x_2 \begin{bmatrix}-4\\1\\0\\0\\0\\0\end{bmatrix}+x_5\begin{bmatrix}0\\0\\0\\0\\1\\0\end{bmatrix}+x_6\begin{bmatrix}-11\\0\\8\\3\\0\\1\end{bmatrix} }
\subsection*{5}
Suppose the solution set of a certain system of linear equations can be described as $x_1=5+3x_3, x_2=-3-6x_3$ with $x_3$ free. Use vectors to describe this set as a line in $\mathbb{R}^3$
Geometrically, the solution set is a line through $\begin{bmatrix}5\\-3\\0\end{bmatrix}$ and parallel to $\begin{bmatrix}3\\-6\\1\end{bmatrix}$
\subsection*{6}
$3x_1 + 3x_2 + 6x_3 = 12,\qquad  \qquad 3x_1 + 3x_2 + 6x_3 = 0,$\\
$-9x_1 - 9x_2 - 18x_3 = -36, \qquad -9x_1 - 9x_2 - 18x_3 = 0,$\\
$-7x_2 + 14x_3 = 14,\quad \qquad \qquad -7x_2 + 14x_3 = 0$\\[0.5in]
Describe the solutions of the first system of equations below in parametric vector form. Provide a geometric comparison with the solution set of the second system of equations below
\\Describe the solution set of the first system of equations in parametric vector form. Select the correct choice below and fill in the answer box(es) within your choice.
$\mathbf{x}=\begin{bmatrix}6\\-2\\0\end{bmatrix}  +x_3 \begin{bmatrix}-4\\2\\1\end{bmatrix}$
\\Which option best compares the two systems? The solution set of the first system is a line parallel to the line that is the solution set of the second system.
\subsection*{7}
Find a parametric equation of the line M through p and q for the given values of p and q. Hint: M is parallel to the vector q-p shown in the figure.
\\$p=\begin{bmatrix}-7\\2\end{bmatrix},q=\begin{bmatrix}0\\-4\end{bmatrix}$
\\Answer:$\mathbf{x}=\begin{bmatrix}0\\-4\end{bmatrix}+t\begin{bmatrix}7\\-6\end{bmatrix}(t \epsilon \mathbb{R})$
\subsection*{8}
A is a $3\times3$ matrix with three pivot positions.
a: Does the equation Ax=0 have a nontrivial solution? No
b: Does the equation Ax=b have at least one solution for every possible b? Yes
\subsection*{9}
A is a $2\times5$ matrix with three pivot positions.
a: Does the equation Ax=0 have a nontrivial solution? Yes
b: Does the equation Ax=b have at least one solution for every possible b? Yes
\subsection*{10}
Given $A=\begin{bmatrix}-2&-6\\5&15\\-3&-9\end{bmatrix},$ find one nontrivial solution of $Ax=0$ by inspection. 
\\$\mathbf{x}=\begin{bmatrix}-3\\1\end{bmatrix}$
\end{document}
