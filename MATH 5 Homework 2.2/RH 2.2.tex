% !root= RH 1.6.tex
\documentclass{article}
\usepackage{tikz}
\usepackage{pgfplots}
\usepackage{setspace}
\usepackage{units}
\usepackage{graphicx}
\usepackage{amsopn}
\usepackage{bbding}
\usepackage{amsmath}
\usepackage{hyperref}
\usepackage{cancel}
\usepackage{etoolbox}
\usepackage{enumitem}
\usepackage{gensymb}
\usepackage{amssymb}
\usepackage{multicol}
\usepackage{numerica}
\usepackage{tikz}
\usetikzlibrary{arrows.meta, calc}
\usepackage[top=0.7in, left=0.8in, right=0.8in, bottom=1in]{geometry}
\AtBeginEnvironment{document}{\everymath{\displaystyle}}
\pagestyle{plain}
\title{RH 1.9}
\author{MATH 5, Jones}
\date{Tejas Patel}
\begin{document}
\maketitle
\section*{Refrigerator Homework}
\subsection*{10}
$AD = I 
\\A^{-1} (AD) = A^{-1} I
\\(A^{-1} A) D = A^{-1} I
\\I D = A^{-1}
\\D = A^{-1}$
\subsection*{24}
$(B - C)D = 0
\\(B - C)D D^{-1} = 0 D^{-1}
\\(B - C)I = 0
\\B - C = 0
\\B=C$
\subsection*{29}
$C^{-1} (A + X) B^{-1} = I_n
\\(A + X) B^{-1} = C I_n = C
\\A + X = CB
\\X = CB - A$
\subsection*{30}

a: B  is invertible because it appears in an equation where both sides represent the inverse of an invertible matrix. If  B  were not invertible, the equation would not properly define an inverse on the left-hand side.
b: $(A - AX)^{-1} = X^{-1} B.
\\A - AX = B^{-1} X.
\\A = AX + B^{-1} X.
\\A = (A + B^{-1})X.
\\X = (A + B^{-1})^{-1} A$
\subsection*{32}
Because if its invertible its linearly independent, meaning it spans $\mathbb{R}^n$. Simple stuff really
\pagebreak \section*{Computer Homework: Next 10 Pages}
\end{document}
