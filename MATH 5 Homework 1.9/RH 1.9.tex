% !root= RH 1.6.tex
\documentclass{article}
\usepackage{tikz}
\usepackage{pgfplots}
\usepackage{setspace}
\usepackage{units}
\usepackage{graphicx}
\usepackage{amsopn}
\usepackage{bbding}
\usepackage{amsmath}
\usepackage{hyperref}
\usepackage{cancel}
\usepackage{etoolbox}
\usepackage{enumitem}
\usepackage{gensymb}
\usepackage{amssymb}
\usepackage{multicol}
\usepackage{numerica}
\usepackage{tikz}
\usetikzlibrary{arrows.meta, calc}
\usepackage[top=0.7in, left=0.8in, right=0.8in, bottom=1in]{geometry}
\AtBeginEnvironment{document}{\everymath{\displaystyle}}
\pagestyle{plain}
\title{RH 2.1}
\author{MATH 5, Jones}
\date{Tejas Patel}
\begin{document}
\maketitle
\section*{Refrigerator Homework}
\subsection*{10}
 \( T: \mathbb{R}^2 \to \mathbb{R}^2 \) first reflects points through the vertical \( x_2 \)-axis and then rotates points \( \frac{3\pi}{2} \) radians.
\\$\begin{bmatrix}
  1&0\\0&1
\end{bmatrix} \rightarrow \text{Reflection through the $x_2$ axis from textbook table} \rightarrow \begin{bmatrix}
  -1&0\\0&1
\end{bmatrix}$
\\[0.1in]$\frac{3\pi}{2}$ rad rotation (90 degree clockwise)$\rightarrow \begin{bmatrix}
  0&1\\-1&0
\end{bmatrix}$
\\[0.1in]Multiplying them together: 
$\begin{bmatrix}-1&0\\0&1\end{bmatrix} \begin{bmatrix}0&1\\-1&0\end{bmatrix}= 
\begin{bmatrix}0&1\\-1&0\end{bmatrix}=$\boxed{\begin{bmatrix} 0 & -1 \\ -1 & 0 \end{bmatrix}}
\subsection*{17}
Show that \( T \) is a linear transformation by finding a matrix that implements the mapping. Note that \( x_1, x_2, \dots \) are not vectors but are entries in vectors.
\\\( T(x_1, x_2, x_3, x_4) = (0, x_1 + x_2, x_2 + x_3, x_3 + x_4) \)
\\T = $\begin{bmatrix}
  0 & 0 & 0 & 0 \\
  1 & 1 & 0 & 0 \\
  0 & 1 & 1 & 0 \\
  0 & 0 & 1 & 1
  \end{bmatrix}$ Since all values inside the matrix are real and make the transformation linear and nothing \\[0.05in]other than linear, the transformation can be considered linear
\subsection*{23}
True. The transformation matrix is calculated using the identity matrix, meaning\\Identitiy Matrix $\rightarrow$ Transformation Matrix
\subsection*{25}
Yes. Rotations are linear transformations as they scale by 1 and dont have any nonlinear effect on the original point
\subsection*{33}
$
A = \begin{bmatrix}
0 & 0 & 0 & 0 \\
1 & 1 & 0 & 0 \\
0 & 1 & 1 & 0 \\
0 & 0 & 1 & 1
\end{bmatrix}
$ only has 3 pivots so it only maps into $\mathbb{R}^3$, not $\mathbb{R}^4$ so it is neither onto and also means its not one-to-one since there's a row of zeores
\subsection*{35}
$
T \begin{bmatrix} x_1 \\ x_2 \\ x_3 \end{bmatrix} =
\begin{bmatrix}
1 & -5 & 4 \\
0 & 1 & -6
\end{bmatrix}
\begin{bmatrix} x_1 \\ x_2 \\ x_3 \end{bmatrix}
$ makes the transformation matrix $
A = \begin{bmatrix}
1 & -5 & 4 \\
0 & 1 & -6
\end{bmatrix}
$. It is onto since there are 2 pivots, one for each row, but is not one-to-one since its not true that the only element of each row is the pivot
\subsection*{43}
For $\mathbb{R}^n \rightarrow \mathbb{R}^m$ A linear transformation  T  is onto if its image spans the entire codomain  $\mathbb{R}^m$ , which means the rank of the transformation matrix  A  must be equal to  m  (i.e., the number of linearly independent rows must be  $m$). That wasy $n\geq m$
\\A linear transformation  T  is one-to-one if the null space of  A  is trivial, meaning the only solution to  Ax = 0  is the zero vector. This occurs when  A  has full column rank, meaning: $n \leq m$
\\ For  T  to be both onto ($ n \geq m $) and one-to-one ($ n \leq m $), we require: $n = m$

\pagebreak \section*{Computer Homework: Next 10 Pages}
\end{document}
