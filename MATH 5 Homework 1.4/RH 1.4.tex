% !root= RH 1.3.tex
\documentclass{article}
\usepackage{tikz}
\usepackage{pgfplots}
\usepackage{setspace}
\usepackage{units}
\usepackage{graphicx}
\usepackage{amsopn}
\usepackage{bbding}
\usepackage{amsmath}
\usepackage{hyperref}
\usepackage{cancel}
\usepackage{etoolbox}
\usepackage{enumitem}
\usepackage{gensymb}
\usepackage{amssymb}
\usepackage{multicol}
\usepackage{numerica}
\usepackage[top=1in, left=0.8in, right=0.8in, bottom=1in]{geometry}
\AtBeginEnvironment{document}{\everymath{\displaystyle}}
\pagestyle{plain}
\title{RH 1.3}
\author{MATH 5, Jones}
\date{Tejas Patel}
\begin{document}
\maketitle
\section{Refrigerator Homework}
\subsection*{Practice Problem 3}
Let \textbf{$w_1, w_2, w_3, u, v$} be vectors in $\mathbb{R}^n$. Suppose the vectors \textbf{u} and \textbf{v} are in Span ${w_1, w_2, w_3}$
g. Show that $u + v$ is also in Span ${w_1, w_2, w_3}$

Since $u$ and $v$ exist in Span ${w_1, w_2, w_3}$, there must bc constants $x_n$ for $u$ and $y_n$ for $v$ where $u= \sum_{m=1}^{n} x_mw_m$ and $v= \sum_{m=1}^{n} y_mw_m$
. With this, $u+v=x_1w_1+x_2w_2+x_3w_3+y_1w_1+y_2w_2+y_3w_3= (y_1+x_1)w_1+(y_2+x_2)w_2+(y_3+x_3)w_3$
\\Since all three of those coefficients are scalar quantities, \boxed{u+v$ is in Span ${w_1, w_2, w_3}}
\subsection*{17}
$a_1=\left[\begin{array}{c}
    1\\4\\-2
\end{array}\right]a_2=\left[\begin{array}{c}
-2\\-3\\7
\end{array}\right]b=\left[\begin{array}{c}
4\\1\\h
\end{array}\right]$ for what value(s) of $h$ is $b$ in the plane spanned by $a_1$ and $a_2$?
\\[0.1in]$a_1-2a_2=4 \qquad 4a_1-3a_2=1$\\$5a_2=-15 \rightarrow a_2=-3$\\$a_1+6=4 \rightarrow a_1=-2$\\$-2(-2)+-3(7)=h \rightarrow$ \boxed{h=-17}
\subsection*{21}
$ -2c_1 + 2c_2 = h $  \\
$ c_1 + c_2 = k $  \\
Second equation:  \\
$ c_1 = k - c_2 $  \\
Substituting into the first equation:  \\
$ -2(k - c_2) + 2c_2 = h \rightarrow -2k + 2c_2 + 2c_2 = h $  \\
$ -2k + 4c_2 = h \rightarrow c_2 = \frac{h + 2k}{4} $  \\
Substituting back:  \\
$ c_1 = k - \frac{h + 2k}{4} = \frac{4k - h - 2k}{4} $  
\\$ c_1 = \frac{2k - h}{4} $ \\ 
Since division by 0 never occurs and the system always has solutions for any $ h $ and $ k $, it follows that the span of $ \mathbf{u} $ and $ \mathbf{v} $ covers the entire $ \mathbb{R}^2 $, meaning they form a basis for $ \mathbb{R}^2 $.  
\subsection*{23}
T/F $\left[\begin{array}{c}
    -4\\3
\end{array}\right]\equiv \left[-4 \; 3\right]$ is \textbf{false}. Row vectors and column vectors have different meanings in the way they are read and are only equivalent when marked as transposed
\subsection*{25}
T/F (-2,5) and (-5,2) lie on a line passing through the origin. This statement is \textbf{false} as the line equation is $y=x+7$ and (0,0 is not a solution)
\subsection*{33}
a: No, 3
\\b: 
$
\begin{bmatrix}
1 & 0 & -4 & 4 \\
0 & 3 & -2 & 1 \\
-2 & 6 & 3 & -4
\end{bmatrix}
$
\\$R_3+=2R_1 \rightarrow \begin{bmatrix}
    1 & 0 & -4 & 4 \\
    0 & 3 & -2 & 1 \\
    0 & 6 & -5 & 4
    \end{bmatrix}$ \\[0.1in]$R_3-=2R_2 \rightarrow
    \begin{bmatrix}
        1 & 0 & -4 & 4 \\
        0 & 3 & -2 & 1 \\
        0 & 0 & 1 & 2
        \end{bmatrix}
    $\\[0.1in]$R_1+=4R_3$ and $R_2+=2R_3 \rightarrow
    \begin{bmatrix}
        1 & 0 & 0 & 12 \\
        0 & 3 & 0 & 5 \\
        0 & 0 & 1 & 2
        \end{bmatrix}
    $ \\$R_2 /= 3 \rightarrow \begin{bmatrix}
        1 & 0 & 0 & 12 \\
        0 & 1 & 0 & 5/3 \\
        0 & 0 & 1 & 2
        \end{bmatrix}$ Yes \textbf{b} is in $W$ and there are an infinite number of vectors in W
\\c: $a_1=1\cdot a_1+ 0a_2 + 0a_3$ and $x_1=1, x_2=0,x_3=0$ \checkmark \pagebreak
\section*{Computer Homework}
\subsection*{1}
Write a vector equation that is equivalent to the given system of equations.
$\\x_2+2x_3=0 \\ 4x_1+65x_2-x_3=0 \\ -x_1+6x_2-8x_3=0$
\\Answer: 
$x_1 \begin{array}{c}0\\4\\-1\end{array} + x_2 \begin{array}{c}1\\6\\6\end{array} + x_3 \begin{array}{c}2\\-1\\-8\end{array}= \begin{array}{c}0\\0\\0\end{array}$
\subsection*{2}
Determine if b is a linear combination of $a_1, a_2, a_3$ .
\[\mathbf{a}_1 =\begin{bmatrix}1 \\-3 \\0\end{bmatrix},\quad\mathbf{a}_2 =\begin{bmatrix}0 \\1 \\4\end{bmatrix},\quad\mathbf{a}_3 =\begin{bmatrix}5 \\-6 \\36\end{bmatrix},\quad\mathbf{b} =\begin{bmatrix}4 \\-3 \\36\end{bmatrix}\]
Vector $b$ is a linear combination of a $a_1,a_2,a_3$. The pivots in the corresponding echelon matrix are in the first entry in the first column and the second entry in the second column
\subsection*{3}
Determine if b is a linear combination of the vectors formed from the columns of the matrix A.
\[\mathbf{A} =\left[\begin{array}{ccc}1 & -3 & 5 \\0 & 5 & 2 \\-4 & 12 & -20\end{array}\right],\quad\mathbf{b} =\left[\begin{array}{c}2 \\-6 \\-2\end{array}\right]\]
Vector $b$ is not a linear combination of the vectors formed from the columns of the matrix A.
\subsection*{4}
Determine if b is a linear combination of the vectors formed from the columns of the matrix A.
\\[0.1in]$ \mathbf{A} = \left[ \begin{array}{ccc} 1 & -3 & -5 \\ 0 & 7 & 6 \\ 4 & -12 & 19 \end{array} \right], \quad \mathbf{b} = \left[ \begin{array}{c} 12 \\ -7 \\ 6 \end{array} \right] $
\\Vector $\mathbf{ b}$ is a linear combination of the vectors formed from the columns of the matrix A. The pivots in the corresponding echelon matrix are in the first entry in the first column, the second entry in the second column, and the third entry in the third column.
\subsection*{5}
Give a geometric description of Span $\{\mathbf{v}_1, \mathbf{v}_2\}$ for the vectors 
$\mathbf{v}_1 = \begin{bmatrix} 4 \\ 6 \\ -8 \end{bmatrix}$ and 
$\mathbf{v}_2 = \begin{bmatrix} 8 \\ 12 \\ -16 \end{bmatrix}.$
$\text{Span} \{\mathbf{v}_1, \mathbf{v}_2\}$ is the set of points on the line through $\mathbf{v}_1$ and $\mathbf{0}$.
\subsection*{6}
\[
\text{Let } A = \begin{bmatrix} 1 & 0 & -2 \\ 0 & 3 & -4 \\ -6 & 9 & 3 \end{bmatrix} 
\text{ and } \mathbf{b} = \begin{bmatrix} 3 \\ -4 \\ -36 \end{bmatrix}.
\]
Denote the columns of \( A \) by \( \mathbf{a}_1, \mathbf{a}_2, \mathbf{a}_3 \), and let \( W = \text{Span} \{\mathbf{a}_1, \mathbf{a}_2, \mathbf{a}_3\} \).

\begin{enumerate}
    \item[(a)] Is \( \mathbf{b} \) in \( \{\mathbf{a}_1, \mathbf{a}_2, \mathbf{a}_3\} \)? How many vectors are in \( \{\mathbf{a}_1, \mathbf{a}_2, \mathbf{a}_3\} \)?
    \item[(b)] Is \( \mathbf{b} \) in \( W \)? How many vectors are in \( W \)?
    \item[(c)] Show that \( \mathbf{a}_1 \) is in \( W \). \textit{[Hint: Row operations are unnecessary.]}
\end{enumerate}
a: No, $\mathbf{b}$ is not in $\{\mathbf{a}_1, \mathbf{a}_2, \mathbf{a}_3\}$ since $\mathbf{b}$ is not equal to $\mathbf{a}_1$, $\mathbf{a}_2$, or $\mathbf{a}_3$. There is(are) 3 vector(s) in \{$a_1, a_2, a_3$\}. \\
b: Yes, B is in $W$ since $b=-a_1-4a_2-2a_3$ and there are infinitely many vecotrs in W \\
c: The vector $a_1$ can be written as the linear combination\{$ a_1+0a_2+0a_3$\}
\subsection*{7}
A mining company has two mines. One day's operation at mine \#1 produces ore that contains 40 metric tons of copper and 510 kilograms of silver, 
while one day's operation at mine \#2 produces ore that contains 50 metric tons of copper and 470 kilograms of silver. 
Let \( \mathbf{v}_1 = \begin{bmatrix} 40 \\ 510 \end{bmatrix} \) and \( \mathbf{v}_2 = \begin{bmatrix} 50 \\ 470 \end{bmatrix} \).
Then \( \mathbf{v}_1 \) and \( \mathbf{v}_2 \) represent the "output per day" of mine \#1 and mine \#2, respectively.
a: What physical interpretation can be given to the vector \( 3\mathbf{v}_1 \)?
\textbf{It is the output at mine \#1 after 3 days of operation.} \(\checkmark\)
\\b. Suppose the company operates mine \#1 for \( x_1 \) days and mine \#2 for \( x_2 \) days. Write a vector equation in terms of \( \mathbf{v}_1 \) and \( \mathbf{v}_2 \) whose solution gives the number of days each mine should operate in order to produce 240 tons of copper and 2591 kilograms of silver. Do not solve the equation.
\[x_1 \mathbf{v}_1 + x_2 \mathbf{v}_2 = \begin{bmatrix} 240 \\ 2591 
\end{bmatrix}
\]
c: Solve.
Mine 1: 2.5 days, Mine 2: 2.8 days


\subsection*{8}
Construct a $3\times3$ matrix A, with nonzero entries, and$ a$ vector $b$ in set of real numbers $\mathbb{R}^3$ such that $b$ is not in the set spanned by the columns of A.
\\Answer: A=$
\left[\begin{array}{ccc}
    1&1&1\\2&2&2\\3&3&3
\end{array}\right]$ and B= $\left[\begin{array}{c}
    4\\5\\6
\end{array}\right]$
\subsection*{9}
Determine whether the statement below is true or false. Justify the answer.
The points in the plane corresponding to $\left[\begin{array}{c}
    -2\\5
\end{array}\right]$ and $\left[\begin{array}{c}
    -5\\2
\end{array}\right]$ lie on a line through the origin

The statement is false. If $\left[\begin{array}{c}
    -2\\5
\end{array}\right]$ and $\left[\begin{array}{c}
    -5\\2
\end{array}\right]$ were on a line then theyd be mutiples of each other, which is not the case
\subsection*{10}
Determine whether the statement below is true or false. Justify the answer.
\\An example of a linear combination of vectors $v_1$ and $v_2$   is the vector $\frac{1}{2}v_1$
\\True. $\frac{1}{2} \mathbf{v}_1 = \frac{1}{2} \mathbf{v}_1 + 0 \mathbf{v}_2$
\end{document}