% !root= RH 1.3.tex
\documentclass{article}
\usepackage{tikz}
\usepackage{pgfplots}
\usepackage{setspace}
\usepackage{units}
\usepackage{graphicx}
\usepackage{amsopn}
\usepackage{bbding}
\usepackage{amsmath}
\usepackage{hyperref}
\usepackage{cancel}
\usepackage{etoolbox}
\usepackage{enumitem}
\usepackage{gensymb}
\usepackage{amssymb}
\usepackage{multicol}
\usepackage{numerica}
\usepackage[top=1in, left=0.8in, right=0.8in, bottom=1in]{geometry}
\AtBeginEnvironment{document}{\everymath{\displaystyle}}
\pagestyle{plain}
\title{RH 1.4}
\author{MATH 5, Jones}
\date{Tejas Patel}
\begin{document}
\maketitle
\section*{Refrigerator Homework}
\subsection*{13}
$\begin{bmatrix}
    3&-5\\-2&6\\1&1
\end{bmatrix}\begin{bmatrix}
    x_1\\x_2\\x_3
\end{bmatrix} = \begin{bmatrix}
    0\\4\\4
\end{bmatrix}$ becomes the system 
$\begin{bmatrix}
    3&-5&0\\-2&6&4\\1&1&4
\end{bmatrix}$ and can be solved using row reduction
\\[0.1in] $R_2+=2R_3\rightarrow \begin{bmatrix}
    3&-5&0\\0&8&12\\1&1&4
\end{bmatrix}$
\\[0.1in] $R_1-=3R_3 \rightarrow
\begin{bmatrix}
    0&-8&-12\\0&8&12\\1&1&4
\end{bmatrix}$
\\[0.1in]$ R_1 += R_2 \rightarrow \begin{bmatrix}
    0&0&0\\0&8&12\\1&1&4
\end{bmatrix}$
\\[0.1in] $R_2 /= 8 \rightarrow \begin{bmatrix}
    0&0&0\\0&1&1.5\\1&1&4
\end{bmatrix}$ From here, $X_2=1.5, X_1+1.5=4$, so $X_1=2.5$
\\[0.1in]Answer: \boxed{$Yes, and the solution is $ X_1=2.5, \quad X_2=1.5}
\subsection*{15}
Part a: \textbf{Counterexample: $b_1=0,\quad b_2=1$}
\\Part b: $\begin{bmatrix}
    2&-1&b_1\\-6&3&b_2
\end{bmatrix}\rightarrow R_1 *=-3 \rightarrow \begin{bmatrix}
    -6&3&-3b_1\\-6&3&b_2
\end{bmatrix}$ \\[0.1in]this shows the system is consistent for \boxed{$all possibilities where $b_2=-3b_1}\pagebreak
\subsection*{18}
$\begin{bmatrix}1 & 3 & -2 & 2 \\0 & 1 & 1 & -5 \\1 & 2 & -3 & 7 \\-2 & -8 & 2 & -1\end{bmatrix} \rightarrow R_1-=R_3 \; \& \; R_4+=2R_3\rightarrow 
\begin{bmatrix}0 & 1 & 1 & -5 \\0 & 1 & 1 & -5 \\1 & 2 & -3 & 7 \\0 & -4 & -4 & 13\end{bmatrix} \rightarrow R_2-=R_1\rightarrow
\begin{bmatrix}0 & 1 & 1 & -5 \\0 & 0 & 0 & 0 \\1 & 2 & -3 & 7 \\0 & -4 & -4 & 13\end{bmatrix}$
\\[0.1in]By Theorem 1.4, since there does not exist a pivot in all 4 rows, it it not possible for matrix $B$ to span $\mathbb{R}^4$
\subsection*{20}
Part a: No, also by Theorem 1.4, since $B$ does not span $\mathbb{R}^4$, not all vectors in $\mathbb{R}^4$ can be written as a linear combination of the colum of B
\\[0.1in]Part b: $\begin{bmatrix}0 & 1 & 1 & -5 \\0 & 0 & 0 & 0 \\1 & 2 & -3 & 7 \\0 & -4 & -4 & 13\end{bmatrix}\rightarrow R_4 += 4R_1 \rightarrow 
\begin{bmatrix}0 & 1 & 1 & -5 \\0 & 0 & 0 & 0 \\1 & 2 & -3 & 7 \\0 & 0 & 0 & -7\end{bmatrix} \rightarrow R_4 /=-7 \rightarrow \begin{bmatrix}0 & 1 & 1 & -5 \\0 & 0 & 0 & 0 \\1 & 2 & -3 & 7 \\0 & 0 & 0 & 1\end{bmatrix} \\[0.1in]
R_3 -=2R_1 \rightarrow \begin{bmatrix}0 & 1 & 1 & -5 \\0 & 0 & 0 & 0 \\1 & 0 & -5 & 3 \\0 & 0 & 0 & 1\end{bmatrix} \rightarrow R_3 -= 3R_4 \; \& \; R_1+=5R_4 \rightarrow \begin{bmatrix}0 & 1 & 1 & 0 \\0 & 0 & 0 & 0 \\1 & 0 & -5 & 0 \\0 & 0 & 0 & 1\end{bmatrix}$
\\[0.1in] Rearrange the rows: $\begin{bmatrix}1 & 0 & -5 & 0 \\0 & 1 & 1 & 0 \\0 & 0 & 0 & 1\\0 & 0 & 0 & 0 \end{bmatrix}$ 
\\[0.1in] No, it does not span all of $\mathbb{R}^3$ and as it spans a 3 dimensional subspace of $\mathbb{R}^4$.
\subsection*{23}
False. It's a Matrix Equation. That's the title of this section.
\subsection*{32}
True. Distributing the $\mathbf{x}$ out and tacking on the b into the end of the new matrix, you will end up with a system that is in the form of an augmented matrix
\subsection*{33}
True. By Theorem 1.4, if it is inconsistent for any $b$ then it is not true that there is a pivot in every row.
\subsection*{43}
For a 4 row $\times$ 3 column matrix to have a unique solution for all $b$, there will be 4 equations and only 3 variables for them to link, meaning if all the variables are solved then there will be one row of zeroes, and there will have been more equations than necessary.
This is because only one variable can take up a row when a unique solution is solved and there are rows that won't contain a solved variable. Since all 3 variables are solved for, this means all of the variables can be scaled to different amounts and the solutions to the matrix span
all of $\mathbb{R}^3$ \pagebreak
\section*{Computer Homework}
\subsection*{1}
Compute the product using (a) the definition where $A\mathbf{x}$ is the linear combination of the columns of A using the corresponding entries in $\mathbf{x}$ as weights, and (b) the row-vector rule for computing $A\mathbf{x}$. If a product is undefined, explain why.
$\qquad \begin{bmatrix}
    -3&8\\3&4\\0&4
\end{bmatrix}
\begin{bmatrix}
6\\-6\\5
\end{bmatrix}$
\\[0.1in]a: The matrix-vector $A\mathbf{x}$ is not defined because the number of columns in matrix $A$ does not match the number of entries in the vector $\mathbf{x}$.
\\b: 
The matrix-vector $A\mathbf{x}$ is not defined because the row-vector rule states that the number of columns in matrix $A$ must match the number of entries in the vector $\mathbf{x}$.
\subsection*{2}
Use the definition of $A\mathbf{x}$ to write the matrix equation as a vector equation.
$\begin{bmatrix} 0 & 8 \\ 6 & 0 \\ 2 & -3 \\ -6 & 6 \end{bmatrix}\begin{bmatrix} 3 \\ 5 \end{bmatrix}=\begin{bmatrix} 40 \\ 18 \\ -9 \\ 12 \end{bmatrix}$
\\The vector equation is: $3\begin{bmatrix}
    0\\6\\2\\-6
\end{bmatrix}+5\begin{bmatrix}
    8\\0\\-3\\6
\end{bmatrix}=\begin{bmatrix}
    40\\18\\-9\\12
\end{bmatrix}$
\subsection*{3}
Write the system first as a vector equation and then as a matrix equation: \\$5x_1+x_2-3x_3=1 \qquad 9x_2+3x_3=0$
\\[0.1in] Vector Equation: $x_1 \begin{bmatrix}5\\0\end{bmatrix} +x_2 \begin{bmatrix}1\\9\end{bmatrix} + x_3 \begin{bmatrix}-3 \\3\end{bmatrix}=\begin{bmatrix}1\\0\end{bmatrix}$
\\[0.1in] Matrix Equation: $\begin{bmatrix}5&1&-3\\0&9&3\end{bmatrix}\begin{bmatrix}x_1\\x_2\\x_3\end{bmatrix} = \begin{bmatrix}1\\0\end{bmatrix}$
\subsection*{4}
Given A and $\mathbf{b}$ to the right, write the augmented matrix for the linear system that corresponds to the matrix equation A$\mathbf{x}$=$\mathbf{b}$. Then solve the system and write the solution as a vector.
A$=\begin{bmatrix}1&3&-1\\1&5&3\\4&2&4\end{bmatrix},b=\begin{bmatrix}-5\\23\\-20\end{bmatrix}$
\\[0.1in]Augmented Matrix: $\begin{bmatrix}1&3&-1&-5\\1&5&3&23\\4&2&4&-20\end{bmatrix}$
\\[0.1in]System solution as a vector: $\begin{bmatrix}-12\\4\\5\end{bmatrix}$ \pagebreak
\subsection*{5}
Let A=$\begin{bmatrix}-1&&2\\4&&-8\end{bmatrix}$ and $\mathbf{b}=\begin{bmatrix}b_1\\b_2\end{bmatrix}$. Show that the equation A$\mathbf{x}$=$\mathbf{b}$ does not have a solution for some choices of $\mathbf{b}$, \\ [0.1in]and describe the set of all $\mathbf{b}$ for which A$\mathbf{x}$=$b$ does have a solution.
\\\textbf{a:} How can it be shown that the equation A$\mathbf{x}$=$\mathbf{b}$ does not have a solution for some choices of $\mathbf{b}$?
\\Row reduce the matrix A to demonstrate that A does not have a pivot position in every row.
\\\textbf{b:} Describe the set of all $\mathbf{b}$ for which A$\mathbf{x}$=$b$ does have a solution.
\\The set of all $\mathbf{b}$ for which A$\mathbf{x}$=$b$ does have a solution is the set of solutions to the equation 
$0=4b_1+b_2$
\subsection*{6}
Let $\mathbf=\begin{bmatrix}5\\5\\20\end{bmatrix}$ and $A=\begin{bmatrix}4&-6\\-3&7\\2&2\end{bmatrix}$. Is $\mathbf{u}$ in the plane $\mathbb{R}^3$ spanned by the columns of A? Why or why not?
\\[0.1in]Yes, multiplying A by the vector $\begin{bmatrix}6.5\\3.5\end{bmatrix}$ writes $\mathbf{u}$ as a linear combination of the columns of A.
\subsection*{7}
Do the columns of  A span $\mathbb{R}^4$? Does the equation $A\mathbf{x} = \mathbf{b}$ have a solution for each $\mathbf{b}$ in $\mathbb{R}^4$?
$A = \begin{bmatrix}1 & 4 & 10 & -5 \\0 & 3 & 9 & -6 \\3 & 4 & 6 & 1 \\-3 & -8 & -18 & 2\end{bmatrix}$
\\a: Do the columns of  A span $\mathbb{R}^4$?
\\No, because the reduced echelon form of A is $\begin{bmatrix}1&0&-2&0\\0&1&3&0\\0&0&0&1\\0&0&0&0\end{bmatrix}$

\subsection*{8}
Let \( \mathbf{v}_1 = \begin{bmatrix} 0 \\ 0 \\ -3 \end{bmatrix} \),  
\( \mathbf{v}_2 = \begin{bmatrix} 0 \\ -5 \\ -9 \end{bmatrix} \), and  
\( \mathbf{v}_3 = \begin{bmatrix} 4 \\ -4 \\ -6 \end{bmatrix} \).  

Does \( \{ \mathbf{v}_1, \mathbf{v}_2, \mathbf{v}_3 \} \) span \( \mathbb{R}^3 \)? Why or why not?  

\textbf{Choose the correct answer below.}  

\textbf{A.} Yes. When the given vectors are written as the columns of a matrix \( A \), \( A \) has a pivot position in every row.  
\subsection*{9}
\textbf{Determine whether the statement below is true or false. Justify the answer.}
Every matrix equation \( A \mathbf{x} = \mathbf{b} \) corresponds to a vector equation with the same solution set.
\textbf{Choose the correct answer below.}
\textbf{A.} This statement is true. The matrix equation \( A \mathbf{x} = \mathbf{b} \) is simply another notation for the vector equation:
\[x_1 \mathbf{a}_1 + x_2 \mathbf{a}_2 + \cdots + x_n \mathbf{a}_n = \mathbf{b}\]
where \( \mathbf{a}_1, \dots, \mathbf{a}_n \) are the columns of \( A \).
\subsection*{10}
Suppose \( A \) is a \( 4 \times 3 \) matrix and \( \mathbf{b} \) is a vector in \( \mathbb{R}^4 \) with the property that \( A \mathbf{x} = \mathbf{b} \) has a unique solution. What can you say about the reduced echelon form of \( A \)? Justify your answer.
\textbf{C.} The first 3 rows will have a pivot position and the last row will be all zeros. If a row had more than one 1, then there would be an infinite number of solutions for \( a_m x_m = b_m \).
\end{document}