% !root= RH 1.3.tex
\documentclass{article}
\usepackage{tikz}
\usepackage{pgfplots}
\usepackage{setspace}
\usepackage{units}
\usepackage{graphicx}
\usepackage{amsopn}
\usepackage{bbding}
\usepackage{amsmath}
\usepackage{hyperref}
\usepackage{cancel}
\usepackage{etoolbox}
\usepackage{enumitem}
\usepackage{gensymb}
\usepackage{amssymb}
\usepackage{multicol}
\usepackage{numerica}
\usepackage[top=1in, left=0.8in, right=0.8in, bottom=1in]{geometry}
\AtBeginEnvironment{document}{\everymath{\displaystyle}}
\pagestyle{plain}
\title{RH 1.4}
\author{MATH 5, Jones}
\date{Tejas Patel}
\begin{document}
\maketitle
\section*{Refrigerator Homework}
\subsection*{13}
$\begin{bmatrix}
    3&-5\\-2&6\\1&1
\end{bmatrix}\begin{bmatrix}
    x_1\\x_2\\x_3
\end{bmatrix} = \begin{bmatrix}
    0\\4\\4
\end{bmatrix}$ becomes the system 
$\begin{bmatrix}
    3&-5&0\\-2&6&4\\1&1&4
\end{bmatrix}$ and can be solved using row reduction
\\[0.1in] $R_2+=2R_3\rightarrow \begin{bmatrix}
    3&-5&0\\0&8&12\\1&1&4
\end{bmatrix}$
\\[0.1in] $R_1-=3R_3 \rightarrow
\begin{bmatrix}
    0&-8&-12\\0&8&12\\1&1&4
\end{bmatrix}$
\\[0.1in]$ R_1 += R_2 \rightarrow \begin{bmatrix}
    0&0&0\\0&8&12\\1&1&4
\end{bmatrix}$
\\[0.1in] $R_2 /= 8 \rightarrow \begin{bmatrix}
    0&0&0\\0&1&1.5\\1&1&4
\end{bmatrix}$ From here, $X_2=1.5, X_1+1.5=4$, so $X_1=2.5$
\\[0.1in]Answer: \boxed{$Yes, and the solution is $ X_1=2.5, \quad X_2=1.5}
\subsection*{15}
Part a: \textbf{Counterexample: $b_1=0,\quad b_2=1$}
\\Part b: $\begin{bmatrix}
    2&-1&b_1\\-6&3&b_2
\end{bmatrix}\rightarrow R_1 *=-3 \rightarrow \begin{bmatrix}
    -6&3&-3b_1\\-6&3&b_2
\end{bmatrix}$ \\[0.1in]this shows the system is consistent for \boxed{$all possibilities where $b_2=-3b_1}\pagebreak
\subsection*{18}
$\begin{bmatrix}1 & 3 & -2 & 2 \\0 & 1 & 1 & -5 \\1 & 2 & -3 & 7 \\-2 & -8 & 2 & -1\end{bmatrix} \rightarrow R_1-=R_3 \; \& \; R_4+=2R_3\rightarrow 
\begin{bmatrix}0 & 1 & 1 & -5 \\0 & 1 & 1 & -5 \\1 & 2 & -3 & 7 \\0 & -4 & -4 & 13\end{bmatrix} \rightarrow R_2-=R_1\rightarrow
\begin{bmatrix}0 & 1 & 1 & -5 \\0 & 0 & 0 & 0 \\1 & 2 & -3 & 7 \\0 & -4 & -4 & 13\end{bmatrix}$
\\[0.1in]By Theorem 1.4, since there is no pivot in all 4 rows, it it not possible for matrix $B$ to span $\mathbb{R}^4$
\subsection*{20}
Part a: No, also by Theorem 1.4, since $B$ does not span $\mathbb{R}^4$, not all vectors in $\mathbb{R}^4$ can be written as a linear combination of the colum of B
\\[0.1in]Part b: $\begin{bmatrix}0 & 1 & 1 & -5 \\0 & 0 & 0 & 0 \\1 & 2 & -3 & 7 \\0 & -4 & -4 & 13\end{bmatrix}\rightarrow R_4 += 4R_1 \rightarrow 
\begin{bmatrix}0 & 1 & 1 & -5 \\0 & 0 & 0 & 0 \\1 & 2 & -3 & 7 \\0 & 0 & 0 & -7\end{bmatrix} \rightarrow R_4 /=-7 \rightarrow \begin{bmatrix}0 & 1 & 1 & -5 \\0 & 0 & 0 & 0 \\1 & 2 & -3 & 7 \\0 & 0 & 0 & 1\end{bmatrix} \\[0.1in]
R_3 -=2R_1 \rightarrow \begin{bmatrix}0 & 1 & 1 & -5 \\0 & 0 & 0 & 0 \\1 & 0 & -5 & 3 \\0 & 0 & 0 & 1\end{bmatrix} \rightarrow R_3 -= 3R_4 \; \& \; R_1+=5R_4 \rightarrow \begin{bmatrix}0 & 1 & 1 & 0 \\0 & 0 & 0 & 0 \\1 & 0 & -5 & 0 \\0 & 0 & 0 & 1\end{bmatrix}$
\\[0.1in] Rearrange the rows: $\begin{bmatrix}1 & 0 & -5 & 0 \\0 & 1 & 1 & 0 \\0 & 0 & 0 & 1\\0 & 0 & 0 & 0 \end{bmatrix}$ 
\\[0.1in] No, it does not span all of $\mathbb{R}^3$ and the counterexample is $\{ 5, -1, 1, 0 \}$
\subsection*{23}
False. It's a Matrix Equation. That's the title of this section.
\subsection*{32}
True. Distributing the $\mathbf{x}$ out and tacking on the b into the end of the new matrix, you will end up with a system that is in the form of an augmented matrix
\subsection*{33}
True. By Theorem 1.4, if it is inconsistent for any $b$ then it is not true that there is a pivot in every row.
\subsection*{43}
\section*{Computer Homework}

\end{document}