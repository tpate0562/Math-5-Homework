% !root= RH 1.2.tex
\documentclass{article}
\usepackage{tikz}
\usepackage{pgfplots}
\usepackage{setspace}
\usepackage{units}
\usepackage{graphicx}
\usepackage{amsopn}
\usepackage{bbding}
\usepackage{amsmath}
\usepackage{hyperref}
\usepackage{cancel}
\usepackage{etoolbox}
\usepackage{enumitem}
\usepackage{gensymb}
\usepackage{amssymb}
\usepackage{multicol}
\usepackage{numerica}
\usepackage[top=1in, left=0.8in, right=0.8in, bottom=1in]{geometry}
\AtBeginEnvironment{document}{\everymath{\displaystyle}}
\pagestyle{plain}
\title{RH 1.2}
\author{MATH 5, Jones}
\date{Tejas Patel}
\begin{document}
\maketitle
\section{Refrigerator Homework}
\subsection*{3: Convert matrix to RREF}
$
\left[\begin{array}{ccc|c}
    1 & 2 & 3 & 4 \\
    4 & 5 & 6 & 7 \\
    6 & 7 & 8 & 9 
\end{array}\right]
$ Subtract $4R_1$ from $R_2$ and $6R_1$ from $R_3 \rightarrow
\left[\begin{array}{ccc|c}
    1 & 2 & 3 & 4 \\
    0 & -3 & -6 & -9 \\
    0 & -5 & -10 & -15 
\end{array}\right]
$\\[0.1in]Scale $R_2$ by $-\frac{1}{3}$ and $R_3$ by $-\frac{1}{5} \rightarrow 
\left[\begin{array}{ccc|c}
    1 & 2 & 3 & 4 \\
    0 & 1 & 2 & 3 \\
    0 & 1 & 2 & 3 
\end{array}\right]$ \\[0.1in] Subtract $R_2$ from $R_3 \rightarrow
\left[\begin{array}{ccc|c}
    1 & 2 & 3 & 4 \\
    0 & 1 & 2 & 3 \\
    0 & 0 & 0 & 0 
\end{array}\right]$ \\[0.1in] Subtract $2R_2$ from $R_1 \rightarrow
\boxed{\left[\begin{array}{ccc|c}
    1 & 0 & -1 & -2 \\
    0 & 1 & 2 & 3 \\
    0 & 0 & 0 & 0 
\end{array}\right]}
$ is the resultant matrix in RREF
\subsection*{7: Find the general solution to the system}
$
\left[\begin{array}{ccc|c}
    1 & 3 & 4 & 7 \\
    3 & 9 & 7 & 6
\end{array}\right]
\rightarrow$ Subtract $3R_1$ from $R_2 \rightarrow$
$
\left[\begin{array}{ccc|c}
    1 & 3 & 4 & 7 \\
    0 & 0 & -5 & -15
\end{array}\right]$ \\[0.1in] Scale $R_2$ by $-\frac{1}{5} \rightarrow
\left[\begin{array}{ccc|c}
    1 & 3 & 4 & 7 \\
    0 & 0 & 1 & 3
\end{array}\right]
$\\[0.1in] Subtract $4R_2$ from $R_1 \rightarrow
\left[\begin{array}{ccc|c}
    1 & 3 & 0 & -5 \\
    0 & 0 & 1 & 3
\end{array}\right]
$ \\[0.1in] Free variable: $X_2 = t$
\\$X_1=-5-3t \\ X_2 = t \\ X_3=3 \\ \boxed{(-5-3t, t, 3)}$ \pagebreak
\subsection*{14: Find the general solution to the system}
$\left[\begin{array}{ccccc|c}
    1 & 2 & -5 & -4 & 0 & -5 \\
    0 & 1 & -6 & -4 & 0 & 2 \\
    0 & 0 & 0 & 0 & 1 & 0 \\
    0 & 0 & 0 & 0 & 0 & 0 
\end{array}\right] \rightarrow$ Subtract $2R_2$ from $R_1 \rightarrow
\left[\begin{array}{ccccc|c}
    1 & 0 & 7 & 4 & 0 & -9 \\
    0 & 1 & -6 & -4 & 0 & 2 \\
    0 & 0 & 0 & 0 & 1 & 0 \\
    0 & 0 & 0 & 0 & 0 & 0 
\end{array}\right]
$ \\[0.1in] Free Variables $X_3 = s, \; X_4=t$
\\Solution: $X_1 = -9-7s-4t, \; X_2 = 2+6s+4t, \; X_3 = s, \; X_4 = t, \; X_5 = 0$
\\[1.2mm]\boxed{ -9-7s-4t,\; 2+6s+4t, \; s, \; t, \;  0}
\subsection*{21: Find h where the system is consistent}
$\left[\begin{array}{cc|c}
    2 & 3 & h \\
    4 & 6 & 7 
\end{array}\right] \rightarrow$ Scale $R_1$ by $2 \rightarrow
\left[\begin{array}{cc|c}
    4 & 6 & 2h \\
    4 & 6 & 7 
\end{array}\right] \rightarrow 2h=7 \rightarrow \boxed{h=3.5}
$
\subsection*{24: Choose h \& k such that the system has (a) no solution, (b) a unique solution, and (c) many solutions}
$\left[\begin{array}{cc|c}
    1 & 3 & 2 \\
    3 & h & k 
\end{array}\right]$
\subsubsection*{a}
At $h=9$ and $k\neq 6$, you'd need to scale $R_1$ by 3 and show $6\neq 6$ meaning there would be no solution to the system
\subsubsection*{b}
At $h=4$ and $k=3$ there is one solution, as long as $h \neq 9$ and $k \neq 6$ there will be one solution. In this case the solution is $X_1=\frac{1}{5} \qquad X_2 = \frac{3}{5}$
\subsubsection*{c}
At $h=9$ and $k=6$ the two equations will be scalar multiples of each other, meaning multiple solutions exist.
\subsection*{35: Suppose a $3\times 5$ \textit{coefficient} matrix for a system has three pivot columns. Is the system consistent? Why or why not?}
3 Rows $\times$ 5 Columns means there will be 5 variables and only 3 equations to relate the variables to each other. Since there are 5 variables but 3 equations and 3 pivot points, we know there are 2 free variables, which means \boxed{$the system could have an infinite solution set, meaning it is consistent$}
\subsection*{37: Suppose the coefficient matrix of a system of linear equations has a pivot position in every row. Explain why the system is consistent.}
If there is a pivot in every row, that means every row is defined to equal the constant it shares a row with. Since every variable is mapped to a constant, there
 \boxed{$is a unique solution and the system is consistent$}\\
\pagebreak
\section*{Computer Homework}
\subsection*{1: Convert matrix to RREF}
$\left[\begin{array}{ccc|c}
    1 & 2 & 3 & 4 \\
    3& 4 & 5 & 6 \\
    6 & 7 & 8 & 9 
\end{array}\right]
\rightarrow$ Subtract $6R_1$ from $R_3$ and $3R_1$ from $R_2$ $\rightarrow
\left[\begin{array}{ccc|c}
    1 & 2 & 3 & 4 \\
    0 & -2 & -4 & -6 \\
    0 & -5 & -10 & -15 
\end{array}\right]$ \\[0.1in] Subtract $2.5R_2$ from $R_3 \rightarrow
\left[\begin{array}{ccc|c}
    1 & 2 & 3 & 4 \\
    0 & -2 & -4 & -6 \\
    0 & 0 & 0 & 0 
\end{array}\right]$ \\[0.1in] Scale $R_2$ by $-\frac{1}{2} \rightarrow
\left[\begin{array}{ccc|c}
    1 & 2 & 3 & 4 \\
    0 & 1 & 2 & 3 \\
    0 & 0 & 0 & 0 
\end{array}\right]
$\\[0.1in] Subtract $R_2$ form $R_1\rightarrow
\boxed{\left[\begin{array}{ccc|c}
    1 & 0 & -1 & -2 \\
    0 & 1 & 2 & 3 \\
    0 & 0 & 0 & 0 
\end{array}\right]}
$The matrix is now in RREF with pivot columns 1 and 2
\subsection*{2: Select the }
\subsection*{3: Find the general solution to the matrix}
$\left[\begin{array}{ccc|c}
1&4&4&16\\
2&8&3&7
\end{array}\right] \rightarrow R_2 \mathrel{-}= 2R_1\rightarrow 
\left[\begin{array}{ccc|c}
1&4&4&16\\
0&0&-5&-25
\end{array}\right]\rightarrow  R_2 \mathrel{/}= -5\rightarrow
\left[\begin{array}{ccc|c}
1&4&4&16\\
0&0&1&5
\end{array}\right]\\[3mm]
\left[\begin{array}{ccc|c}
1&4&4&16\\
0&0&1&5
\end{array}\right] \rightarrow R_1 \mathrel{-}=4R_2 \rightarrow
\left[\begin{array}{ccc|c}
1&4&0&-4\\
0&0&1&5
\end{array}\right] 
$\\[3mm] $X_2 = t$ is a free variable, meaning the solution set is: \boxed{(-4-4t, t, 5)}
\subsection*{4: Find the general solution to the matrix}

$\left[\begin{array}{ccc|c}
    0&1&-5&6\\
    1&-3&13&-12
    \end{array}\right] \rightarrow R_2 \mathrel{-}=3R_1 \rightarrow
    \left[\begin{array}{ccc|c}
    0&1&-5&6\\
    1&0&-2&6
    \end{array}\right] \rightarrow$ Swap rows $\rightarrow
    \left[\begin{array}{ccc|c}
    1&0&-2&6\\
    0&1&-5&6
    \end{array}\right]$ \\[3mm] Making $X_3 = t$ a free variable and the solution set \boxed{(6+2t, 6+5t, t)}
\subsection*{5: Find the general solution to the matrix}
$\left[\begin{array}{ccc|c}
    5&-3&7&0\\
    10&-6&14&0\\ 
    15&-9&21&0
\end{array}\right]\rightarrow \text{Divide } R_2 \text{ by 2 and } R_3 \text{ by 3} \rightarrow
\left[\begin{array}{ccc|c}
    5&-3&7&0\\
    5&-3&7&0\\
    5&-3&7&0
\end{array}\right]$ making $X_2=s$ and $X_3=t$ free\\[0.1in] variables and the solution to the system \boxed{\left(\frac{3}{5}s-\frac{7}{5}t, s, t\right)}
\pagebreak
\subsection*{6: Find the general solution to the matrix}
$
\left[\begin{array}{cccccc}
    1 & -4 & 0 & -1 & 0 & -9 \\
    0 & 1 & 0 & 0 & -5 & 3 \\
    0 & 0 & 0 & 1 & 6 & 7  \\
    0 & 0 & 0 & 0 & 0 & 0
\end{array}\right] \rightarrow
R_1 \mathrel{+}= 4R_2 \rightarrow 
\left[\begin{array}{cccccc}
    1 & 0 & 0 & -1 & -20 & 3 \\
    0 & 1 & 0 & 0 & -5 & 3 \\
    0 & 0 & 0 & 1 & 6 & 7  \\
    0 & 0 & 0 & 0 & 0 & 0
\end{array}\right] \\[3mm]
R_1 \mathrel{+}= R_3 \rightarrow 
\left[\begin{array}{cccccc}
    1 & 0 & 0 & 0 & -14 & 10 \\
    0 & 1 & 0 & 0 & -5 & 3 \\
    0 & 0 & 0 & 1 & 6 & 7  \\
    0 & 0 & 0 & 0 & 0 & 0
\end{array}\right]
$
\\[0.1in] Variables $X_3=s, \; X_5=t$ is are free variables, and the solution to the system is $\boxed{10+14t, 3+5t, s, 7-6t, t}$
\subsection*{7: Choose h and k such that the system has (a) no solution, (b) a unique solution, and (c) many solutions.}
System:
$\begin{cases}
    x_1+hx_2=5\\
    5x_1+ 15x_2=k
\end{cases}$
\\a: No solutions when $h=3$ and $k \neq 25$
\\b: Unique solution when $h\neq3$
\\c: Many solutions when $h=3$ and $k=25$
\subsection*{8: Suppose a 3$\times$6 coefficient matrix for a system has three pivot columns. Is the system consistent? Why or why not?}
The augmented matrix will have seven columns and will not have a row of the form $\left[\begin{array}{cccccc|c}
    0&0&0&0&0&0&1
\end{array}\right]$
, so the system is consistent. This is because every row has a pivot, so there will be no blank coefficient rows
\subsection*{9: Suppose a system of linear equations has a 3$\times$5 augmented matrix whose fifth column is not a pivot column. Is the system consistent? Why or why not?}
A linear system is consistent if and only if the rightmost column of the augmented matrix is not a pivot column. That is, if and only if an echelon form of the augmented matrix has no row of the form [0 ... 0 b] with b nonzero.
\\In the augmented matrix described above, is the rightmost column a pivot column? No
\\ In the echelon form of the augmented matrix, is there a row of the form [0 0 0 0 b] with b nonzero? No
\\ \boxed{$Therefore, by the Existence and Uniqueness Theorem, the linear system is consistent.$}
\subsection*{10: Suppose the coefficient matrix of a system of linear equations has a pivot position in every row. Explain why the system is consistent.}
\boxed{$The system is consistent because the rightmost column of the augmented matrix is not a pivot column. $}
\end{document}