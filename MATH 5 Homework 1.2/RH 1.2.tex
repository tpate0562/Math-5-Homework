% !root= RH 1.2.tex
\documentclass{article}
\usepackage{tikz}
\usepackage{pgfplots}
\usepackage{setspace}
\usepackage{units}
\usepackage{graphicx}
\usepackage{amsopn}
\usepackage{bbding}
\usepackage{amsmath}
\usepackage{hyperref}
\usepackage{cancel}
\usepackage{etoolbox}
\usepackage{enumitem}
\usepackage{gensymb}
\usepackage{amssymb}
\usepackage{multicol}
\usepackage{numerica}
\usepackage[top=1in, left=0.8in, right=0.8in, bottom=1in]{geometry}
\AtBeginEnvironment{document}{\everymath{\displaystyle}}
\pagestyle{plain}
\title{RH 1.2}
\author{MATH 5, Jones}
\date{Tejas Patel}
\begin{document}
\maketitle
\subsection*{3: Convert matrix to RREF}
$
\left[\begin{array}{ccc|c}
    1 & 2 & 3 & 4 \\
    4 & 5 & 6 & 7 \\
    6 & 7 & 8 & 9 
\end{array}\right]
$ Subtract $4R_1$ from $R_2$ and $6R_1$ from $R_3 \rightarrow
\left[\begin{array}{ccc|c}
    1 & 2 & 3 & 4 \\
    0 & -3 & -6 & -9 \\
    0 & -5 & -10 & -15 
\end{array}\right]
$\\[0.1in]Scale $R_2$ by $-\frac{1}{3}$ and $R_3$ by $-\frac{1}{5} \rightarrow 
\left[\begin{array}{ccc|c}
    1 & 2 & 3 & 4 \\
    0 & 1 & 2 & 3 \\
    0 & 1 & 2 & 3 
\end{array}\right]$ \\[0.1in] Subtract $R_2$ from $R_3 \rightarrow
\left[\begin{array}{ccc|c}
    1 & 2 & 3 & 4 \\
    0 & 1 & 2 & 3 \\
    0 & 0 & 0 & 0 
\end{array}\right]$ \\[0.1in] Subtract $2R_2$ from $R_1 \rightarrow
\boxed{\left[\begin{array}{ccc|c}
    1 & 0 & -1 & -2 \\
    0 & 1 & 2 & 3 \\
    0 & 0 & 0 & 0 
\end{array}\right]}
$ is the resultant matrix in RREF
\subsection*{7: Find the general solution to the system}
$
\left[\begin{array}{ccc|c}
    1 & 3 & 4 & 7 \\
    3 & 9 & 7 & 6
\end{array}\right]
\rightarrow$ Subtract $3R_1$ from $R_2 \rightarrow$
$
\left[\begin{array}{ccc|c}
    1 & 3 & 4 & 7 \\
    0 & 0 & -5 & -15
\end{array}\right]$ \\[0.1in] Scale $R_2$ by $-\frac{1}{5} \rightarrow
\left[\begin{array}{ccc|c}
    1 & 3 & 4 & 7 \\
    0 & 0 & 1 & 3
\end{array}\right]
$\\[0.1in] Subtract $4R_2$ from $R_1 \rightarrow
\left[\begin{array}{ccc|c}
    1 & 3 & 0 & -5 \\
    0 & 0 & 1 & 3
\end{array}\right]
$ \\[0.1in] Free variable: $X_2 = t$
\\$X_1=-5-3t \\ X_2 = t \\ X_3=3 \\ \boxed{(-5-3t, t, 3)}$ 
\subsection*{14: Find the general solution to the system}
$\left[\begin{array}{ccccc|c}
    1 & 2 & -5 & -4 & 0 & -5 \\
    0 & 1 & -6 & -4 & 0 & 2 \\
    0 & 0 & 0 & 0 & 1 & 0 \\
    0 & 0 & 0 & 0 & 0 & 0 
\end{array}\right] \rightarrow$ Subtract $2R_2$ from $R_1 \rightarrow
\left[\begin{array}{ccccc|c}
    1 & 0 & 7 & 4 & 0 & -9 \\
    0 & 1 & -6 & -4 & 0 & 2 \\
    0 & 0 & 0 & 0 & 1 & 0 \\
    0 & 0 & 0 & 0 & 0 & 0 
\end{array}\right]
$ \\[0.1in] Free Variables $X_3 = s, \; X_4=t$
\\Solution: $X_1 = -9-7s-4t, \; X_2 = 2+6s+4t, \; X_3 = s, \; X_4 = t, \; X_5 = 0$
\\[1.2mm]\boxed{ -9-7s-4t,\; 2+6s+4t, \; s, \; t, \;  0} \pagebreak
\subsection*{21: Find h where the system is consistent}
$\left[\begin{array}{cc|c}
    2 & 3 & h \\
    4 & 6 & 7 
\end{array}\right] \rightarrow$ Scale $R_1$ by $2 \rightarrow
\left[\begin{array}{cc|c}
    4 & 6 & 2h \\
    4 & 6 & 7 
\end{array}\right] \rightarrow 2h=7 \rightarrow \boxed{h=3.5}
$
\subsection*{24: Choose h \& k such that the system has (a) no solution, (b) a unique solution, and (c) many solutions}
$\left[\begin{array}{cc|c}
    1 & 3 & 2 \\
    3 & h & k 
\end{array}\right]$
\subsubsection*{a}
At $h=9$ and $k\neq 6$, you'd need to scale $R_1$ by 3 and show $6\neq 6$ meaning there would be no solution to the system
\subsubsection*{b}
At $h=4$ and $k=3$ there is one solution, as long as $h \neq 9$ and $k \neq 6$ there will be one solution. In this case the solution is $X_1=\frac{1}{5} \qquad X_2 = \frac{3}{5}$
\subsubsection*{c}
At $h=9$ and $k=6$ the two equations will be scalar multiples of each other, meaning multiple solutions exist.
\subsection*{35: Suppose a $3\times 5$ \textit{coefficient} matrix for a system has three pivot columns. Is the system consistent? Why or why not?}
\subsection*{37: Suppose the coefficient matrix of a system of linear equations has a pivot position in every row. Explain why the system is consistent.}





\end{document}