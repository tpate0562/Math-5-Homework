% !root= RH 1.6.tex
\documentclass{article}
\usepackage{tikz}
\usepackage{pgfplots}
\usepackage{setspace}
\usepackage{units}
\usepackage{graphicx}
\usepackage{amsopn}
\usepackage{bbding}
\usepackage{amsmath}
\usepackage{hyperref}
\usepackage{cancel}
\usepackage{etoolbox}
\usepackage{enumitem}
\usepackage{gensymb}
\usepackage{amssymb}
\usepackage{multicol}
\usepackage{numerica}
\usepackage{tikz}
\usetikzlibrary{arrows.meta, calc}
\usepackage[top=0.7in, left=0.8in, right=0.8in, bottom=1in]{geometry}
\AtBeginEnvironment{document}{\everymath{\displaystyle}}
\pagestyle{plain}
\title{RH 1.10}
\author{MATH 5, Jones}
\date{Tejas Patel}
\begin{document}
\maketitle
\section*{Refrigerator Homework}
\subsection*{6}
$\begin{cases}
  5I_1 + 2I_2 = 30 \\
  2I_1 + 7I_2 + 4I_3 = 20 \\
  4I_2 + 9I_3 + I_4 = 40 \\
  3I_1 + I_3 + 6I_4 = 10
  \end{cases} = \begin{bmatrix}
    5&2&0&0&30\\2&7&4&0&20\\0&4&9&1&40\\0&0&2&-7&10
  \end{bmatrix}\rightarrow \text{RREF} \rightarrow \begin{bmatrix} 1 & 0 & 0 & 0 & \frac{2002}{291} \\[0.1in] 0 & 1 & 0 & 0 & -\frac{640}{291} \\[0.1in] 0 & 0 & 1 & 0 & \frac{1574}{291} \\[0.1in] 0 & 0 & 0 & 1 & \frac{34}{291} \end{bmatrix}$ All quantities in SI Amperes
\subsection*{9}
$C_{n+1} = 0.93C_n + 0.05S_n \\ 
S_{n+1} = 0.95S_n + 0.07C_n$
\\If $C_0=800,000$ and $S_0=500,000$, $C_2=741720$, $S_2=558280$ 
\subsection*{12}

$P = \begin{bmatrix}
  0.97 & 0.05 & 0.10 \\
  0.00 & 0.90 & 0.05 \\
  0.03 & 0.05 & 0.85
  \end{bmatrix} \qquad 
  x_0 = \begin{bmatrix} 295 \\ 55 \\ 150 \end{bmatrix} \\[0.1in] 
$ If $  x_{n+1} = P x_n, \qquad x_3=\begin{bmatrix}
  311.543\\  58.255\\ 130.202
\end{bmatrix}$

  
   

\end{document}
