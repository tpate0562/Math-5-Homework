% !root= RH 1.6.tex
\documentclass{article}
\usepackage{tikz}
\usepackage{pgfplots}
\usepackage{setspace}
\usepackage{units}
\usepackage{graphicx}
\usepackage{amsopn}
\usepackage{bbding}
\usepackage{amsmath}
\usepackage{hyperref}
\usepackage{cancel}
\usepackage{etoolbox}
\usepackage{enumitem}
\usepackage{gensymb}
\usepackage{amssymb}
\usepackage{multicol}
\usepackage{numerica}
\usepackage{tikz}
\usetikzlibrary{arrows.meta, calc}
\usepackage[top=0.7in, left=0.8in, right=0.8in, bottom=1in]{geometry}
\AtBeginEnvironment{document}{\everymath{\displaystyle}}
\pagestyle{plain}
\title{RH 2.3}
\author{MATH 5, Jones}
\date{Tejas Patel}
\begin{document}
\maketitle
\section*{Refrigerator Homework}
\subsection*{15}
Counterexample: $A=\begin{bmatrix}1&2\\2&4\end{bmatrix}$ where $ad-bc=0$
\subsection*{24}
Suppose A is a 5x5 matrix whose columns do not span $\mathbb{R}^5$. This means that the columns of A are linearly dependent, so at least one of column can be written as a linear combination of the others.
If the columns are linearly dependent, then the equation $Ax=0$ has a nontrivial solution, meaning there exists a nonzero vector x such that $Ax=0$.
However, if $A$ were invertible, the only solution to $Ax=0$ would be $x=0$, which contradicts the existence of a nontrivial solution. Thus, $A$ cannot be invertible if its columns do not span $\mathbb{R}^5$.
\subsection*{34}
Since the columns of $A$ are linearly independent, one of the statements of the Invertible Matrix Theorem tells us that $A$ is invertible (i.e., A has an inverse). This means that $A$ creates a one-to-one and onto transformation from $\mathbb{R}^n$ to $\mathbb{R}^n$. Now, consider $A^2$, which is just $A$ multiplied by itself. Because A is invertible, the product $A^2$ is also invertible. Applying the Invertible Matrix Theorem again, since $A^2$ is invertible, its columns must also be linearly independent and, importantly, they span $\mathbb{R}^n$.
\subsection*{42}
$T(x_1,x_2)=\begin{bmatrix}6x_1-8x_2\\-5x_1+7x_2\end{bmatrix} = \begin{bmatrix}6&-8\\-5&7\end{bmatrix}\begin{bmatrix}x_1\\x_2\end{bmatrix} = -\frac{1}{4} \begin{bmatrix}6&-8\\-5&7\end{bmatrix}^{-1}=\begin{bmatrix}\frac{7}{2} & 4 \\[0.1in] \frac{5}{2} & 3\end{bmatrix}$
\\$T^{-1}=\left(\frac{7}{2}x_1+4x_2, \frac{5}{2}x_1+3x_2\right)$
\pagebreak
\end{document}
